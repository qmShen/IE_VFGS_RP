\subsection{User Study}\label{sec:user}

We conduct a comprehensive user study to evaluate the performance of visualization generated by different methods.

\subsubsection{Settings}
We recruited ** participants with ** females, ** males, aged **-** with a mean of ** for the user study.

We used the trajectory datasets of \pt{} to generate the visualizations.
In addition to $\qtavats$, we consider anther two methods: uniform random sampling $\rand$ and distance based sampling $\baseline$ as the alternatives.
We manually select 22 target points distributed across the Porto and define three visualization scales including:
large scale region(zoom level less than 13), middle scale region(zoom level between 13 and 15), small scale region(zoom level more than 15).
With a given target point and a visualization scale,
we generate a comparable visualization group including 4 visualizations which are generated by full dataset($\full$), uniform random sampling($\rand$), distance based sampling($\baseline$) and our proposal($\qtavats$).

In total, we have 66 comparable groups(22 target points $\times$ 3 scales) and 264 visualization results(66 comparable groups $\times$ 4 visualizations).
We evaluate visualization from two perspectives: visual clutter and visual quality, thus with a given comparable group, we design three tasks:
T1) order the visualizations in a comparable group from least visual clutter to the most clutter;
T2) order the visualizations in a comparable group from highest visual quality to the lowest visual quality;
T3) select the acceptable visualizations for interactive trajectory exploration and choose the reason for the unselected ones. The reason includes "due to sever visual clutter", "due to the poor visual quality" and "others".
The user study system is a web-based platform, in which all visualizations are displayed with a resolution of 450*300.

\subsubsection{User study procedure}

When the participants enter the user study system, they are first given a brief introduction about the motivation of the study and the tasks.
Followed by the introduction, we include a tutorial (with the correct result) to help the participants to get familiar with the interface and tasks.
For each participant, we randomly select 20 comparable groups and generate 60 tasks in total.
In each task, the four visualizations(without the specifying the methods) from one comparable group are shown in one page.
Participants are required to perform the task T1, T2 and T3 according to the four visualization results.
At last, the participants are interviewed to collect feedback after finishing the study and their answers are saved for result analysis.

\subsubsection{Result analysis}
\QM{Provide the following figures:}

Figure mulit-barchart/stack-barchart
x axis: Ranking 
y axis: less clutter / frequency
bar: method

Figure mulit-barchart/stack-barchart
x axis: Ranking 
y axis: better quality / frequency
bar: method

Figure piechart for not acceptable
Acceptable
Unacceptable by quality
Unacceptable by clutter
Unacceptable by both quality and clutter